\documentclass[a4paper,12pt]{article}
\usepackage[utf8]{inputenc}
\usepackage[brazil]{babel}
\usepackage[parfill]{parskip} % Activate to begin paragraphs with an empty line rather than an indent

\usepackage{graphicx} % support the \includegraphics command and options
\usepackage{url} % support the url references

\author{Alberto Vital Santos de Sousa}

\begin{document}
%================CAPA===================
\pagenumbering{gobble} % turn off page numbering
\begin{center}
\begin{figure}[h]
\centering
\includegraphics[width=3cm]{ufpe.png}
\includegraphics[width=4cm]{cin.png}
\end{figure}
%{\Large
Universidade Federal de Pernambuco\\
Centro de Informática\\
Curso de Bacharelado em Engenharia da Computação
%}
\\[3cm]
\textbf{\huge
AndroidDriller:\\[3mm]
Uma ferramenta de mineração de repositórios Android\\
}
\vfill
\begin{flushleft}
Aluno: Alberto Vital Santos de Sousa\\
Orientador: Leopoldo Motta Teixeira\\[5mm]
\end{flushleft}
Recife, junho de 2018
\end{center}

%================CONTRACAPA===================
\newpage
\begin{center}
Universidade Federal de Pernambuco\\
Centro de Informática\\
Curso de Bacharelado em Engenharia da Computação
%}
\\[3cm]
\textbf{\huge
AndroidDriller:\\[3mm]
Uma ferramenta de mineração de repositórios Android\\
}
\vfill

\begin{flushleft}
{\small
Monografia apresentada ao Centro de Informática (CIn)\\
da Universidade Federal de Pernambuco (UFPE), como requisito\\
parcial para conclusão do Curso de Engenharia da Computação,\\
orientada pelo professor Leopoldo Motta Teixeira.
}
\end{flushleft}
Recife, junho de 2018
\end{center}


%================RESUMO===================
\newpage
\section*{Agradecimentos}


%================RESUMO===================
\newpage
\pagenumbering{arabic} % turn page numbering back on
\section{Resumo}
Utilizando ferramentas capazes de minerar dados sobre repositórios, pesquisadores de engenharia de software têm obtido um melhor conhecimento sobre o processo de desenvolvimento de software, em geral. Este trabalho propõe uma ferramenta capaz de minerar repositórios fazendo uso das características comuns de projetos Android para que se possa realizar uma investigação preliminar de como as mudanças acontecem no desenvolvimento de aplicações para a plataforma. 

%Palavras-chave: Android, Sistemas de controle de versão, Git, Mineração de repositórios

%================ABSTRACT===================
\newpage
\section{Abstract}

%Software Engineering researchers have been using tools capable of mining data from repositories to get a better understanding 
%about the software development process. This work proposes a tool for mining repositories that uses the common caracteristics
%of Android Projects so we can perform a preliminar investigation about how changes occur in the platform apps development.

%================SUMARIO===================
\newpage

\tableofcontents

%================INTRODUÇÃO===================
\newpage
\section{Introdução}

Em engenharia de software, a análise de repositórios tem ajudado pesquisadores a
obter um melhor conhecimento sobre o processo de desenvolvimento de um
software \cite{miningGit}. Com isso, é possível predizer bugs e também analisar o padrão de
desenvolvimento utilizado pelos colaboradores do projeto. Para tal fim, existem
ferramentas capazes de minerar dados sobre repositórios, como por exemplo, o
RepoDriller \cite{repodriller}.\\
\\
Ferramentas mais gerais, como o próprio RepoDriller, servem para analisar
software em vários domínios. No desenvolvimento de aplicações Android, há
diversas particularidades envolvidas. Por exemplo, o arquivo de manifesto contém
informações sobre os vários componentes do projeto. Assim, as principais
alterações feitas no repositório são refletidas em modificações desse arquivo.\\
\\
No estudo de repositórios Android, pesquisadores têm usado ferramentas capazes
de extrair dados dos arquivos do tipo apk \cite{Calciati}, \cite{WhoAdded} e \cite{YLyu}. Para realizar estudos
semelhantes, uma outra abordagem seria utilizar uma ferramenta capaz de
minerar os repositórios das aplicações e extrair dados específicos levando em
consideração não só o arquivo apk mas também as modificações dos arquivos de
manifesto, por exemplo.\\


%================OBJETIVOS===================
\newpage
\section{Objetivos}

Este trabalho tem por objetivo realizar uma investigação preliminar de como as
mudanças acontecem no desenvolvimento de aplicações Android para criar
ferramentas de suporte à evolução de aplicativos, e assim, implementar uma
ferramenta capaz de minerar estes repositórios, aproveitando-se das
características particulares de projetos Android.\\

%=============================
%=      Desenvolvimento      =
%=============================

%================CONCEITOSBASICOS===================
\newpage

\section{Conceitos Básicos}

\subsection{Android}
\subsection{Sistemas de controle de versão}
\subsubsection{Git}
\subsection{Mineração de repositórios}
\subsection{RepoDriller}
\cite{repodriller}

%================Trabalhos Relacionados===================
\newpage
\section{Trabalhos Relacionados / Contexto}
%referencias
Em \cite{Calciati}, \cite{WhoAdded}, \cite{YLyu}, etc. Foram feitos estudos tais que poderiam se aproveitar da ferramenta proposta.
%Gancho com a motivação
%citar como seria o uso do androidDriller nesses estudos.


%================Ferramenta===================
\newpage
\section{AndroidDriller}
%Detalhamento geral da ferramenta

\subsection{Metodologia}

Foi criada uma ferramenta Java que faz uso da API provida pelo RepoDriller para
percorrer os commits do repositório. Neste projeto foram implementadas as
classes capazes de colher dados sobre os componentes específicos de Android e
geração de relatórios em arquivos CSV. Para visualização dos dados, foi
implementado um programa na linguagem Python capaz de produzir gráficos a
partir dos arquivos CSV gerados pela ferramenta.\\

\subsection{Implementação}
%diagrama de classes
Explicação mais detalhada de cada classe implementada
\subsection{Experimento}
Detalhamento do experimento realizado na máquina do Cin
\subsection{Resultados}
Análise dos resultados obtidos

%=============================
%=         Conclusão         =
%=============================
\newpage

\section{Conclusão}
%\subsection{Trabalhos Futuros}
%Extender a ferramenta para analisar também o codigo fonte e valida-lo junto ao manifesto.

%================Bibliografia===================
\newpage
%\bibliographystyle{IEEEtranS}
\bibliographystyle{ieeetr}
\bibliography{lib}

\end{document}